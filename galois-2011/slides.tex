\documentclass{beamer}
\usepackage{listings}
\usepackage{pgfpages}
\pgfpagesuselayout{4 on 1}[a4paper,border shrink=5mm,landscape]

\title{Faster persistent data structures through hashing}
\author{Johan Tibell\\johan.tibell@gmail.com}
\date{2011-02-15}

\begin{document}
\lstset{language=Haskell}

\frame{\titlepage}

\begin{frame}
  \frametitle{Motivating problem: Twitter data analys}

  ``I'm computing a communication graph from Twitter data and then
  scan it daily to allocate social capital to nodes behaving in a good
  karmic manner.  The graph is culled from 100 million tweets and has
  about 3 million nodes.''

  \bigskip
  We need a data structure that is
  \begin{itemize}
  \item fast when used with string keys, and
  \item doesn't use too much memory.
  \end{itemize}
\end{frame}

\begin{frame}
  \frametitle{Persistent maps in Haskell}

  \begin{itemize}
  \item \lstinline!Data.Map! is the most commonly used map type.
  \item It's implemented using size balanced trees.
  \item Keys can be of any typs, as long as values of the type can be
    ordered.
  \end{itemize}
\end{frame}

\begin{frame}
  \frametitle{Real world performance of Data.Map}

  \begin{itemize}
  \item Good in theory: no more than \emph{O(log n)} comparisons.
  \item Not great in practice: up to \emph{O(log n)} comparisons!
  \item Many common types are expensive to compare e.g
    \lstinline!String!, \lstinline!ByteString!, and \lstinline!Text!.
  \item Given a string of length \emph{k}, we need \emph{O(k*log n)}
    comparisons to look up an entry.
  \end{itemize}
\end{frame}

\begin{frame}
  \frametitle{Hash tables}
  \begin{itemize}
  \item Hash tables perform well with string keys: \emph{O(k)}
    amortized lookup time for strings of length \emph{k}.
  \item However, we want persistent maps, not mutable hash tables.
  \end{itemize}
\end{frame}

\begin{frame}
  \frametitle{Milan Straka's idea: Patricia trees as spare arrays}
  \begin{itemize}
  \item We can use hashing without using hash tables!
  \item A Patricia tree implements a persistent, sparse array.
  \item Patricia trees are twice as fast as size balanced trees, but
    only work with \lstinline!Int! keys.
  \item Use hashing to derive an \lstinline!Int! from an arbitrary
    key.
  \end{itemize}
\end{frame}

\begin{frame}
  \frametitle{Implementation tricks}
  \begin{itemize}
  \item Patricia trees implement a spare, persistent array of size
    2\^32 (or 2\^64) TODO:escape.
  \item This makes hash collisions rare.
  \item A linked list is perfectly adequet (sp?) collision resolution
    strategy.
  \end{itemize}
\end{frame}

\begin{frame}[fragile]
  \frametitle{Actual implementation}
  \begin{lstlisting}
data HashMap k v
    = Nil
    | Tip {-# UNPACK #-} !Hash
          {-# UNPACK #-} !(FL.FullList k v)
    | Bin {-# UNPACK #-} !Prefix
          {-# UNPACK #-} !Mask
          !(HashMap k v) !(HashMap k v)

type Prefix = Int
type Mask   = Int
type Hash   = Int

data FullList k v = FL !k !v !(List k v)
data List k v = Nil | Cons !k !v !(List k v)
  \end{lstlisting}
\end{frame}

\begin{frame}
  \frametitle{Saving the wor(l)d}
  \begin{itemize}
  \item \lstinline!List k v! uses 2 fewer words per key/value pair
    than \lstinline![(k, v)]!.
  \item \lstinline!FullList! can be unpacked into the \lstinline!Tip!
    constructor as it's not a sum type (unlike \lstinline!List!)
  \item Always unpack word sized types like \lstinline!Int!, unless
    you really need them to be lazy.
  \end{itemize}
\end{frame}

\begin{frame}
  \frametitle{Benchmarks}
\end{frame}

\begin{frame}
  \frametitle{Borrowing from our neighbours}
  \begin{itemize}
  \item Clojure uses a \emph{hash array mapped trie} (HAMT) to
    implement persistent maps.
  \item A HAMT is a...
  \end{itemize}
\end{frame}

\begin{frame}
  \frametitle{Clojure's PersistentHashMap description}
\end{frame}

\begin{frame}[fragile]
  \frametitle{Initial implementation based on ezyang's work}
  TODO: Find real initial implementation
  \begin{lstlisting}
data HashMap k v
    = Empty
    | BitmapIndexed {-# UNPACK #-} !Bitmap
                    {-# UNPACK #-} !(Array (HashMap k v))
    | Leaf {-# UNPACK #-} !(Leaf k v)
    | Full {-# UNPACK #-} !(Array (HashMap k v))
    | Collision {-# UNPACK #-} !Hash
                {-# UNPACK #-} !(Array (Leaf k v))

type Bitmap = Word
  \end{lstlisting}
\end{frame}

\begin{frame}
  \frametitle{Benchmarks}
\end{frame}

\begin{frame}
  \frametitle{Where is all the time spent?}
  Profiling
\end{frame}

\begin{frame}
  \frametitle{A better array copy}
\end{frame}

\begin{frame}
  \frametitle{Benchmarks}
\end{frame}

\begin{frame}
  \frametitle{Possible improvements}
  arraycopy, parallelism
\end{frame}

\begin{frame}
  \frametitle{Aside: the need for a sane hierarchy}
\end{frame}

\begin{frame}
  \frametitle{Summary}
  (Everything I said applies to sets as well)
\end{frame}

\end{document}

%%% Local Variables:
%%% mode: latex
%%% TeX-master: t
%%% TeX-PDF-mode: t
%%% End:
